% !TEX program = xelatex
\documentclass[12pt,hyperref,a4paper,UTF8]{ctexart}
\usepackage{CityUhomework}
\usepackage{booktabs}
\usepackage{bookmark}


%%------------------Beginning of the text-----------------%%
\begin{document}

%%-----------------Cover------------------%%
\cover
\thispagestyle{empty}% Home page does not show page numbers
\newpage


%%-----------------Catalog-------------------%%
\newpage
\tableofcontents

%%-----------------Main text starts here-------------------%%
\newpage

\section{作业}
\subsection{题目1}
\textbf{问:}请简述覆盖与交换技术的特点和区别

\textbf{答:}覆盖技术需要程序员按逻辑把程序分段,覆盖技术允许将程序中不同时刻需要的部分放在内存中;交换技术不要求程序员给出程序段之间的交换结构,交换技术可以将整个进程在内存和外存之间移动,从而在较小的存储空间中运行较多的进程或作业。二者的区别是覆盖发生在运行的程序内部,交换发生在程序之间。

\subsection{题目2}
\textbf{问:}如果内存划分为大小为 100KB、500KB、200 KB、300 KB 和 600 KB(按顺序)的空闲块,则采用首次适应、最佳适应和最坏适应算法各自将如何放置大小分别为 215 KB、414 KB、110 KB 和 430 KB(按顺序)的进程,哪一种算法的内存利用率高?

\textbf{答:}
\begin{enumerate}[I.]
    \item 首次适应算法:215kb放入500kb,414kb放入600kb,110kb放入500kb剩余的285kb中,430kb无法放入,内存不足,内存利用率为$739/1700=43.47\%$
    \item 最佳适应算法:215kb放入300kb,414kb放入500kb,110kb放入200kb,430kb放入600kb,内存利用率为$1169/1700=68.76\%$
    \item 最坏适应算法:215kb放入600kb,414kb放入500kb,110kb放入600kb剩余的385kb中,430kb无法放入,内存不足,内存利用率为$739/1700=43.47\%$
    
    所以最佳适应算法的内存利用率最高

\end{enumerate}

\subsection{题目3}
\textbf{问:}请简述虚拟页式内存管理中,采用快表与单级页表机制时的寻址过程。

\textbf{答:}
\begin{enumerate}[I.]
    \item 快表:CPU访问虚拟地址时,会将地址分为页号和偏移两部分,若TLB命中,则直接通过TLB获取页号,加上偏移量形成物理地址;若TLB未命中,则对应的物理帧号将被更新到TLB中。
    \item 单级页表:首先系统会使用页号访问内存中的页表。页表存储了每个页号到物理页框号的映射。通过页表找到对应的物理页框号,然后将物理页框号与偏移量相加得到物理地址。
\end{enumerate}

\subsection{题目4}
\textbf{问:}某程序执行过程中访问的逻辑地址页号序列为:0,0,1,1,0,3,1,2,2,4,4,3。设页面大小为 100 字节,若该程序的基本可用内存(除程序本身外)为 200 字节,计算采用所学局部置换算法的缺页次数,请结合图说明

\textbf{答:}页面大小为100字节,基本可用内存为200字节,所以可以同时存放两个页面。缺页次数为6次。

\begin{table}[ht!]\label{tab-1}
    \centering
 
    \begin{tabular}{c|c|c|c}
        \toprule
        次数& 访问页 & 内存帧 & 缺页 \\
        \midrule
        1 & 0 & / & 是 \\
        2 & 0 & 0 & 否 \\
        3 & 1 & 0,1 & 是 \\
        4 & 1 & 0,1 & 否 \\
        5 & 0 & 0,1 & 否\\
        6 & 3 & 3,1 & 是 \\
        7 & 1 & 3,1 & 否 \\
        8 & 2 & 3,2 & 是 \\
        9 & 2 & 3,2 & 否 \\
        10 & 4 & 2,4 & 是 \\
        11 & 4 & 2,4 & 否 \\
        12 & 3 & 4,3 & 是 \\
        \bottomrule
    \end{tabular}
    
\end{table}


\end{document}
